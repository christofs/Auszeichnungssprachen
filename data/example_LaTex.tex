\title{Abgeleitete Textformate: Text und Data Mining mit urheberrechtlich geschützten Textbeständen}

\author{
Christof Schöch <schoech@uni-trier.de>\\
Frédéric Döhl <f.doehl@dnb.de>\\
Achim Rettinger <rettinger@uni-trier.de>\\
Evelyn Gius <evelyn.gius@tu-darmstadt.de>\\
Peer Trilcke <trilcke@uni-potsdam.de>\\
Peter Leinen <P.Leinen@dnb.de>\\
Fotis Jannidis <fotis.jannidis@uni-wuerzburg.de>\\
Maria Hinzmann <hinzmannm@uni-trier.de>\\
Jörg Röpke <roepke@uni-trier.de>\\
}

\begin{document}
\maketitle
\section{Zusammenfassung}
Das Text und Data Mining (TDM) mit urheberrechtlich geschützten Texten unterliegt trotz der TDM-Schranke (§60d UrhG) weiterhin Einschränkungen, die u.a. die Speicherung, Veröffentlichung und Nachnutzung der entstehenden Korpora betreffen und das volle Potenzial des TDM in den Digital Humanities ungenutzt lassen. Als Lösung werden abgeleitete Textformate vorgeschlagen: Hier werden urheberrechtlich geschützte Textbestände so transformiert, dass alle wesentlichen urheberrechtlich relevanten Merkmale entfernt werden, verschiedene einschlägige Methoden des TDM aber möglich bleiben. Mehrere abgeleitete Textformate werden aus Sicht der Computational Literary Studies, der Informatik, der Gedächtnisinstitutionen und der Rechtswissenschaften beleuchtet.

\section{1. Einleitung}

Es ist ein offenes Geheimnis in den Digital Humanities (DH), dass es für die Computational Literary Studies (CLS) bezüglich der verfügbaren Textbestände ein \textit{window of opportunity} gibt, das sich um 1800 öffnet und um 1920 wieder schließt. Es öffnet sich erst um 1800, weil für Materialien vor dieser Zeit die technischen Herausforderungen im Bereich Optical Character Recognition (OCR) und Normalisierung von orthographischer Varianz immer noch so groß sind, dass deutlich weniger umfangreiche bzw. qualitativ weniger hochwertige Textsammlungen zur Verfügung stehen als für die Zeit nach 1800. Und es schließt sich um 1920, weil für Materialien, die später erschienen sind, in sehr vielen Fällen (abhängig vom Todesdatum der Autor/innen) das Urheberrecht nach wie vor greift und sowohl das Erstellen als auch das Teilen von Textsammlungen mit Dritten deutlich erschwert. Dieser Umstand hat bedauerlicherweise zur Folge, dass die Setzung von Forschungsschwerpunkten häufig nicht primär von den Erkenntnisinteressen und Zielen der Forschung selbst, sondern wesentlich von technischen und rechtlichen, also dieser Forschung externen Faktoren, bestimmt wird. Als Konsequenz daraus ist eine Forschung auf dem methodischen \textit{state of the art} mit neueren Textbeständen nur begrenzt, teilweise sogar überhaupt nicht möglich. Die Forschung in den CLS verwendet zwar aktuelle, oft aus Informatik, Computerlinguistik und Statistik adaptierte Verfahren, kann sie aber in den meisten Fällen nicht auch auf diejenigen Textbestände anwenden, die unsere zeitgenössische literarische Kultur ausmachen.\footnote{Der vorliegende Beitrag ist im Kontext der Workshopreihe „Strategien für die Nutzbarmachung urheberrechtlich geschützter Textbestände für die Forschung durch Dritte“ entstanden.}

Grundsätzlich kann man vier Ansätze zur Lösung der beschriebenen Problematik unterscheiden: 
\begin{itemize}
\item den Zugang zu lizenzierten Inhalten über eine API;
\item die Nutzung von Analyseplattformen; 
\item die Forschung im \textit{closed room};
\item und die Arbeit mit abgeleiteten Textformaten.
\end{itemize} 

\printbibliography
\end{document}
